\documentclass{article}


\usepackage{arxiv}

\usepackage[utf8]{inputenc} % allow utf-8 input
\usepackage[T1]{fontenc}    % use 8-bit T1 fonts
\usepackage{hyperref}       % hyperlinks
\usepackage{url}            % simple URL typesetting
\usepackage{booktabs}       % professional-quality tables
\usepackage{amsfonts}       % blackboard math symbols
\usepackage{nicefrac}       % compact symbols for 1/2, etc.
\usepackage{microtype}      % microtypography
\usepackage{lipsum}
\usepackage{graphicx}
%\includegraphics{image.png}
\usepackage{listings}
\usepackage{verbatim}

\graphicspath{ {./images/} }


\title{Essentials of Dart}


\author{
 Henry Nguyen \\
  Department of Computer Science\\
  University of Arizona\\
  Tucson, AZ 85721 \\
  \texttt{henryn098@email.arizona.edu} \\
  %% examples of more authors
   \And
 Adam Cunningham \\
  Department of Computer Science\\
  University of Arizona\\
  Tucson, AZ 85721 \\
  \texttt{laser@email.arizona.edu} \\
  \And

  %% \AND
  %% Coauthor \\
  %% Affiliation \\
  %% Address \\
  %% \texttt{email} \\
  %% \And
  %% Coauthor \\
  %% Affiliation \\
  %% Address \\
  %% \texttt{email} \\
  %% \And
  %% Coauthor \\
  %% Affiliation \\
  %% Address \\
  %% \texttt{email} \\
}

\begin{document}
\maketitle
\begin{abstract}

\end{abstract}


% keywords can be removed
%\keywords{First keyword \and Second keyword \and More}


\section{Introduction}

\section{History}
\label{sec:headings}
Dart made its first appearance in 2011. At the time of its creation, Elixir and Kotlin were released the same year. A year later, TypeScript was released by Microsoft. The language was essentially designed to solve the frustrations of JavaScript at the time. Created at Google by Lars Bak and Kasper Lund, Dart is an optionally typed object-oriented language.
can act as a superset of JavaScript with the dart2js compiler. It included optional static type analysis. Fast forward to 2015 and JavaScript received an incredible improvement in ECMAScript 6 (ES6). Something something Flutter increasing popularity and competing with React Native
runs faster than vanilla JavaScript
- Needed to replace JavaScript (before ES6) -> pushed JavaScript to evolve

In 2020, Dart is used for Flutter as it seems to be...

References => https://www.codementor.io/blog/worst-languages-to-learn-3phycr98zk, https://hackernoon.com/why-flutter-uses-dart-dd635a054ebf

Why it matters?
- Flutter is a major project that allows AOT (Ahead of time) and JIT (Just in Time) compilation. This leads to innovative development with hot-reload (reference?) and a smoother UX. Something something 60/120 fps on supported devices

\paragraph{Task modeling.}
We approach this task as a regression problem. For every item and shop pair, we need to predict its next month sales(a number).


\subsection{Headings: second level}
\lipsum[5]
\begin{equation}
\xi _{ij}(t)=P(x_{t}=i,x_{t+1}=j|y,v,w;\theta)= {\frac {\alpha _{i}(t)a^{w_t}_{ij}\beta _{j}(t+1)b^{v_{t+1}}_{j}(y_{t+1})}{\sum _{i=1}^{N} \sum _{j=1}^{N} \alpha _{i}(t)a^{w_t}_{ij}\beta _{j}(t+1)b^{v_{t+1}}_{j}(y_{t+1})}}
\end{equation}

\subsubsection{Headings: third level}
\lipsum[6]

\paragraph{Paragraph}
\lipsum[7]

\section{Control Structures}
\label{sec:others}
\begin{comment}
\lipsum[8] \cite{kour2014real,kour2014fast} and see \cite{hadash2018estimate}.

The documentation for \verb+natbib+ may be found at
\begin{center}
  \url{http://mirrors.ctan.org/macros/latex/contrib/natbib/natnotes.pdf}
\end{center}
Of note is the command \verb+\citet+, which produces citations
appropriate for use in inline text.  For example,
\begin{verbatim}
   \citet{hasselmo} investigated\dots
\end{verbatim}
produces
\begin{quote}
  Hasselmo, et al.\ (1995) investigated\dots
\end{quote}

\begin{center}
  \url{https://www.ctan.org/pkg/booktabs}
\end{center}


\subsection{Figures}
\lipsum[10] 
See Figure \ref{fig:fig1}. Here is how you add footnotes. \footnote{Sample of the first footnote.}
\lipsum[11] 

\begin{figure}
  \centering
  \fbox{\rule[-.5cm]{4cm}{4cm} \rule[-.5cm]{4cm}{0cm}}
  \caption{Sample figure caption.}
  \label{fig:fig1}
\end{figure}

\begin{figure} % picture
    \centering
    \includegraphics{test.png}
\end{figure}

\subsection{Tables}
\lipsum[12]
See awesome Table~\ref{tab:table}.

\begin{table}
 \caption{Sample table title}
  \centering
  \begin{tabular}{lll}
    \toprule
    \multicolumn{2}{c}{Part}                   \\
    \cmidrule(r){1-2}
    Name     & Description     & Size ($\mu$m) \\
    \midrule
    Dendrite & Input terminal  & $\sim$100     \\
    Axon     & Output terminal & $\sim$10      \\
    Soma     & Cell body       & up to $10^6$  \\
    \bottomrule
  \end{tabular}
  \label{tab:table}
\end{table}

\subsection{Lists}
\begin{itemize}
\item Lorem ipsum dolor sit amet
\item consectetur adipiscing elit. 
\item Aliquam dignissim blandit est, in dictum tortor gravida eget. In ac rutrum magna.
\end{itemize}
\end{comment}

\section{Data Types}
\label{sec:headings}
\begin{comment}
We are provided with five datasets from Kaggle: Sales train, Sale test, items, item categories and shops. In the Sales train dataset, it provides the information about the sales’ number of an item in a shop within a day. In the Sales test dataset, it provides the shop id and item id which are the items and shops we need to predict. In the other three datasets, we can get the information about item’s name and its category, and the shops’ name.
\paragraph{Task modeling.}
We approach this task as a regression problem. For every item and shop pair, we need to predict its next month sales(a number).
\paragraph{Construct train and test data.}
In the Sales train dataset, it only provides the sale within one day, but we need to predict the sale of next month. So we sum the day's sale into month's sale group by item, shop, date(within a month).
In the Sales train dataset, it only contains two columns(item id and shop id). Because we need to provide the sales of next month, we add a date column for it, which stand for the date information of next month.
\end{comment}

\section{Subprograms}
\begin{comment}
\label{sec:headings}
We are provided with five datasets from Kaggle: Sales train, Sale test, items, item categories and shops. In the Sales train dataset, it provides the information about the sales’ number of an item in a shop within a day. In the Sales test dataset, it provides the shop id and item id which are the items and shops we need to predict. In the other three datasets, we can get the information about item’s name and its category, and the shops’ name.
\paragraph{Task modeling.}
We approach this task as a regression problem. For every item and shop pair, we need to predict its next month sales(a number).
\paragraph{Construct train and test data.}
In the Sales train dataset, it only provides the sale within one day, but we need to predict the sale of next month. So we sum the day's sale into month's sale group by item, shop, date(within a month).
In the Sales train dataset, it only contains two columns(item id and shop id). Because we need to provide the sales of next month, we add a date column for it, which stand for the date information of next month.
\end{comment}

\section{Summary}
\label{sec:headings}
\begin{comment}
We are provided with five datasets from Kaggle: Sales train, Sale test, items, item categories and shops. In the Sales train dataset, it provides the information about the sales’ number of an item in a shop within a day. In the Sales test dataset, it provides the shop id and item id which are the items and shops we need to predict. In the other three datasets, we can get the information about item’s name and its category, and the shops’ name.
\paragraph{Task modeling.}
We approach this task as a regression problem. For every item and shop pair, we need to predict its next month sales(a number).
\paragraph{Construct train and test data.}
In the Sales train dataset, it only provides the sale within one day, but we need to predict the sale of next month. So we sum the day's sale into month's sale group by item, shop, date(within a month).
In the Sales train dataset, it only contains two columns(item id and shop id). Because we need to provide the sales of next month, we add a date column for it, which stand for the date information of next month.
\end{comment}

\bibliographystyle{unsrt}  
%\bibliography{references}  %%% Remove comment to use the external .bib file (using bibtex).
%%% and comment out the ``thebibliography'' section.


%%% Comment out this section when you \bibliography{references} is enabled.
\begin{thebibliography}{1}
\begin{comment}
\bibitem{kour2014real}
George Kour and Raid Saabne.
\newblock Real-time segmentation of on-line handwritten arabic script.
\newblock In {\em Frontiers in Handwriting Recognition (ICFHR), 2014 14th
  International Conference on}, pages 417--422. IEEE, 2014.

\bibitem{kour2014fast}
George Kour and Raid Saabne.
\newblock Fast classification of handwritten on-line arabic characters.
\newblock In {\em Soft Computing and Pattern Recognition (SoCPaR), 2014 6th
  International Conference of}, pages 312--318. IEEE, 2014.

\bibitem{hadash2018estimate}
Guy Hadash, Einat Kermany, Boaz Carmeli, Ofer Lavi, George Kour, and Alon
  Jacovi.
\newblock Estimate and replace: A novel approach to integrating deep neural
  networks with existing applications.
\newblock {\em arXiv fd arXiv:1804.09028}, 2018.
\end{comment}
\end{thebibliography}
\end{document}
